\documentclass[11pt]{article}
\usepackage[top=1in,right=1in,bottom=1in,left=1in]{geometry}
\usepackage{fancyvrb}
\usepackage{hyperref}
\usepackage{graphicx}
\title{C++ Coding Standards and Style Guide}
\author{The SNO+ Code Integrity Committee\footnote{Contact: P. Jones, \href{p.jones22@physics.ox.ac.uk}{p.jones22@physics.ox.ac.uk}}}
\begin{document}
\maketitle
\begin{abstract}
This document contains coding standards, a set of rules that improve the code performance and stability, and style standards, conventions intended to improve code readability and ease of maintenance.
\end{abstract}

\tableofcontents
\section{Coding Standards}
\subsection{Header Guards}
Multiple inclusions of the the same header will generally cause multiple definitions of the same class, which will cause code to not compile.

To prevent this, a header guard should be used around all code in a declaration (header) file. For example:
\begin{verbatim}
#ifndef __RAT_ClassName__
#define __RAT_ClassName__
   // code here
#endif // __RAT_ClassName__
\end{verbatim}


{\bf Example:} Lines 20, 21, 72 in declaration file example, section \ref{sec:ExampleDec}.

\subsection{Forward declarations}
In cases where the code does not need any information about a class (or struct), but only that it exists, it is not necessary to include the header where that class is defined. Instead, a ``forward declaration'' should be used.

To forward-delcare a type, simply replace ``\#include $<$MyClass.hh$>$'' with a ``class MyClass;'' near the top of the file.

Every file that is included in a declaration file will increase the compile time, therefore it is good practice to reduce the number of header files included.

{\bf Example:} Line 28 in declaration file example, section \ref{sec:ExampleDec}.

\subsection{Namespaces in declaration files}
Statements like ``using namespace X'' should be avoided. Instead, code should be explicitly declared in the correct namespace. For example:
\begin{verbatim}
namespace RAT {
  // code here
}
\end{verbatim}
All RAT code should be declared within the RAT namespace, or in sub namespaces of RAT.

\subsection{Macros and \#define}
C-style preprocessor macros should not used in any SNO+ code, and \#define only used in header guards. Global constants if absolutely required must be given a type and a const prefix, e.g. ``const double eV = 1e-3;'' instead of ``\#define eV 1e-3''.

Macros can be replaced by inline functions, enums and const statements, all of which can be checked by a compiler whilst a macro cannot be. Replacing the C style \#define with const type allows the compiler to type-check the constants.

\subsection{Struct or Class}
In C++, structs are classes with all methods and members public. For consistency with general C++ conventions, structs in RAT should only be used to hold data. That is, they have no methods.

\subsection{Writing a constructor/destructor set}
For classes that have members which are newed (pointers initialized with the ``new'' keyword), a constructor, copy constructor, and destructor set should be written. In the destructor, all ``newed'' pointers must be explicitly deleted with a ``delete'' statement to prevent memory leaks.

\subsection{Inheritance}
Object-oriented design is encouraged in RAT, and when possible a set of related classes should derive from a common base class using public inheritance. In rare cases where \href{http://www.parashift.com/c++-faq-lite/private-inheritance.html}{private or protected inheritance} could be used, a class should instead simply include and instance of the would-be base class as a member.

Classes with any virtual methods must also declare a virtual destructor.

{\bf Example:} Line 33 in declaration file example, section \ref{sec:ExampleDec}.

\subsection{Public, protected or private class members}
All data members in RAT should be private, and protected methods used only required for base classes. In general, classes should minimise their public interface.

RAT code should follow the general C++ ``getter-setter'' convention: member variables are private, and accessible outside the class only by public ``getter'' and ``setter'' methods.

{\bf Example:} Lines 35-57 show public methods and 60-62 protected members in declaration file example, section \ref{sec:ExampleDec}.

\subsection{Pass-by-reference}
In C++, there are two ways of passing variables to functions: pass-by-value and pass-by-reference. Pass-by-value means that an object is handed to the function, and the function makes a copy. This is slow and means that the function cannot modify the value of the variable in the scope from which it was called.

In a pass-by-reference, the function takes as an argument not the object itself, but the memory address of that object. This means that the function acts on that object, not a copy of it, and so can make modifications. It is also much faster, since no copying is necessary. This is equivalent to using pointers, without having to worry about pointers.

RAT functions should pass all variables by reference except for the following exceptions:
\begin{itemize}
\item Basic types, except when their value must be modified, since it is faster to copy e.g. a 32-bit integer than a 64-bit pointer.
\item When a copy/pass by value is required
\end{itemize}

{\bf Example:} Lines 37, 38 in declaration file example, section \ref{sec:ExampleDec}.

\subsection{Using const}
SNO+ code should be fully ``const correct:'' if a local variable is used only as in input, it should be prefixed with the ``const'' keyword. If a method does not alter any of the class members it should be postfixed with const, e.g.
\begin{verbatim}
void GetLambda(const double energy) const;
\end{verbatim}
{\bf Example:} Lines 37, 38 in declaration file example, section \ref{sec:ExampleDec}.

\subsection{Inline functions}
If inline functions are used, they should be kept short: a few of lines long at most.

Inline functions are copied directly into the code where ever they are called. Long inline functions will therefore lead to large code bloats if called many times.

\subsection{Repeated code}
Code should not repeated unnecessarily. Rather, it should placed into a function and called where required.

\subsection{Long functions}
Long functions (beyond a typical display page) are rarely required, and should be avoided.

\subsection{Casting}
SNO+ code should use the C++ casting syntax (e.g. ``static\_cast$<$int$>$(x)'') rather than the C syntax (e.g. ``(int)x'').

This allows casts to be searched for and allows the correct type of cast to be specified by the author.

\subsection{NULL}
When referring to a pointer address, the NULL keyword should be used instead of 0.

\subsection{malloc() and free()}
The malloc and free functions should not be used in SNO+ code. The C++ new and delete statements should be used instead.

\subsection{New and Delete}
Every memory initialization with the ``new'' must be matched with a ``delete,'' except for some ROOT and Geant4 classes which are managed by the external libraries. Note ROOT (prefix T) and Geant4 (prefix G4) classes are memory mananged, but basic types are not e.g. G4double. SNO+ code should default to not deleting ROOT or Geant4 classes.

Failing to delete something that has been new(ed) will cause a memory leak, as the space in memory is allocated but never freed. Attempting to use a pointer that has been deleted is a common cause of neaky bugs and segmentation faults. As such, it is good practice to then set the value of a pointer to NULL after deleting it:
\begin{verbatim}
char* a = new char[10];
delete[] a;
a = NULL;
\end{verbatim}

\subsection{Single Line Conditionals}
The use of single line conditionals (ternary operators) in SNO+ code is discouraged. A single line conditional reads like:
\begin{verbatim}
// ternary expression
a = ( b > c ) ? b : c;

// standard expression
if(b > c)
    a = b;
else
    a = c;
\end{verbatim}
Single line conditionals are notoriously unreadable, and if used should always be accompanied with an explanatory comment.

\subsection{Pre-increment}
Pre-increments should never used in for loops, where they are misleading. Pre-increments here falsely suggest that the loop variable is increased before each loop. For example the for loop below falsely suggests i goes from 1 to 9 inclusive, when indeed it goes from 0 to 9:
\begin{verbatim}
for(int i = 0; i < 10; ++i)
\end{verbatim}

Prefix operators including pre-increment are preferred, however, for non-native types (where relevant). When incrementing a class instance ``A'' (assuming a + operator is defined), ``A++'' makes a temporary copy while ``++A'' does not.

\subsection{Random Numbers and Constants}
Random numbers in RAT should be obtained with the CLHEP random number generator, not ROOT. Only the CLHEP generator is properly initialized with RAT's random seed. Available functions include CLHEP::RandFlat, CLHEP::RandPoisson, CLHEP::RandExponential, and several others.

A new function, RAT::gRandom() will soon be added and will then become the standard generator.

\subsection{ROOT}
ROOT classes should only be used in RAT when necessary, namely for histogramming and data structure operations. The C++ standard library defines many common math operations, object container types, etc.; these should be used whenever possible.

In particular, ROOT object container types such as TClonesArray should be avoided.
\subsection{STL}
The standard template library is exceptionally useful and should be used in SNO+ code whenever possible.
\begin{itemize}
\item Use std::vector$<$T$>$ instead of C style arrays
\item Use std::string instead of char*
\end{itemize}

All strings in the SNO+ code should be STL strings rather than C-style character arrays. They should be copied with C++'s stringstream, and manipulated with STL string functions. Other string function implementations such as those in GLG4StringUtil should be considered deprecated.

\subsection{General Tips}
\begin{itemize}
\item Avoid pointers where possible. Pass by reference is often a better solution.
\item Use double in preference to float.
\item Ask the CIC for comments.
\end{itemize}

\section{Coding Style}
\subsection{File naming}
Within RAT all file names should follow the UpperCamelCase format, while all folder names should be lowercase using underscores as spaces (e.g. ``deapclean\_geo''). Definition (header) files must have the extension .hh and declaration (code) files the extension .cc. Files that declare or define classes should be given the same name as the class.

Within the tools directory the same standard applies, however for ROOT macro files the extension should be .c if the file is executed via a .X command, .cpp if executed via a .L command, or .cc if the file must be compiled via a .L [FileName]+ command.

ratdb file names should be uppercase and match the name of the table within. Large files should be split by index and called TABLENAME\_index.ratdb.

In the python directory, file names are lower case and split by underscores e.g. file\_name.py.

Consistent file naming reduces the time taken to identify a file's purpose. 

\subsection{File Header Comments}
Within RAT all declaration (header) files must have a header doxygen comment as below:
\begin{verbatim}
////////////////////////////////////////////////////////////////////////
/// \class namespace::classname
///
/// \brief   A brief description should go here.
///          It ends with the first blank line.
///
/// \author name <email>
///
/// REVISION HISTORY:\n
///     date : name - change 1 \n
///     date : name - change 2 \n
///
/// \detail A more detailed description goes here.
/// Use as many lines as you like. \n ends a line
///
////////////////////////////////////////////////////////////////////////
\end{verbatim}
Definition (source) files do not need this, as they will have a related declaration (header) file that does.

Source files having no related header file should have the above comment block (except with \textbackslash class switched to \textbackslash file).

The header comment and public interface together should give a user a good idea of what the class does and how to use it.

\subsection{\#include file order and style}
The include statements use the C++ style of angle brackets rather than quotation marks. The include statements themselves are ordered such that CLHEP comes first then in descending order, GEANT4, ROOT, RAT, STL and C headers with a line space between.

The C style quotation marks and C++ angle brackets cause the compiler to search for the file in different ways, the angle brackets is prefered as it will ensure only files on the -I include path are found. A common include order helps understand which part of RAT or external library the file is included from.

\subsection{Class naming}
As with file names the classes follow UpperCamelCase naming.

\subsection{Public, Protected, Private ordering}
Within a class declaration, public methods should come first, followed by protected, then private. Having the methods declared in this order saves time while finding them.

\subsection{Method Ordering}
Within the general ``public, protected, then private'' ordering, RAT classes have their constructor then destructor set first. Methods should then be ordered with constructor/initialiser methods first, then usage methods, then destructive methods. Methods should be declared before members.

\subsection{Function and Method Naming}
Functions and methods should follow UpperCamelCase naming, for consistency with ROOT and GEANT4.

\subsection{Method parameter naming}
Method parameters should follow lowerCamelCase naming.

\subsection{Paramter ordering}
Parameters should be written with input first and output last. For example:
\begin{verbatim}
int MyFunction(int input1, double input2, string &inout, int &output);
\end{verbatim} 

\subsection{Inline functions}
For functions that are one line long, such as ``return x;'', write the function within the class header. If the function, though short, spans multiple lines, write it below the class declaration using the inline keyword:
\begin{verbatim}
class A
{
    double B();
}

inline double A::B()
{
    // code here
}
\end{verbatim}
A large function defined within a class declaration reduces the readability, whereas the inline keyword and definition below the class declaration is much easier to follow. Remember that functions declared inline are copied around to where they are called, leading to a possible trade-off between higher performance and higher memory usage.

\subsection{Basic types}
RAT uses a mixture of standard C++ types, ROOT types, and GEANT4 types where appropriate. Specifically, ROOT types (e.g. Double\_t) are used in the data structure (DS classes), since they are written to disk in ROOT format. GEANT4 types (e.g. G4double) are used in parts of RAT that interact with GEANT4 classes -- geometry, generators, and Gsim. Everywhere else, standard C++ types (e.g. double) should be used.

Additional types with explicit size, such uint32\_t and int64\_t,  are provided in stdint.h. These may be used when variable size is critical (e.g. simulating a 32-bit register in the DAQ), but should otherwise be avoided in favor of traditional integer types (int, long, unsigned, etc.).
\subsection{Class member naming}
Class members should prefixed with a lower case f and then follow UpperCamelCase e.g. fElectronEnergy.

\subsection{Namespace closing}
When closing a namespace add a comment with the format ``// namespace name'' where name is the namespace name. For example:
\begin{verbatim}
namespace RAT
{
 double mouse;
} // namespace RAT
\end{verbatim}

This makes it clear which declarations lie within the namespace.

\subsection{Indenting}
RAT code should be indented based on scope, with tabs turned into four spaces. This improves readability and ensures all editors display the code with the same identation. Tabs should be avoided, since different editors will handle them differently. Most editors can be set to replace the tab character with spaces.

\subsection{Commenting Code}
Only the C++ style // and not the C style /* ... */ should be used in RAT, since the latter cannot be commented around. Class methods should be preceded by a doxygen comment line starting with ///, and class members should be followed by a doxygen ///$<$ comment:
\begin{verbatim}
class A
{
/// Integrate the distribution with the Rorschach-Holtzman method
double InkblotIntegral();

double fAnswer; /// <The answer to life, the universe, and everything
}
\end{verbatim}

Note that //! causes ROOT to ignore the member when building the dictionary.



\section{Example Code Declaration File}
\label{sec:ExampleDec}
{\bf File:} src/geo/PMT/BucketConstructor.hh
\begin{Verbatim}[gobble=0,numbers=left]
////////////////////////////////////////////////////////////////////////
/// \class RAT::BucketConstructor
///
/// \brief   Class that contructs the Bucket Logical Volume using \n
///          the bucket parameters
///
/// \author  Phil Jones <p.jones22@physics.ox.ac.uk>
///
/// REVISION HISTORY:\n
///     07/09 : P.Jones - First Revision, new file. \n
///
/// \detail  Takes the bucket parameters and constructs a bucket \n
///          solid and places it into a logical volume
///
////////////////////////////////////////////////////////////////////////


#ifndef __RAT_BucketConstructor__
#define __RAT_BucketConstructor__

#include <RAT/BucketConstructorParams.hh>
#include <RAT/ConstructorBase.hh>

#include <string>

class G4LogicalVolume;

namespace RAT
{

class BucketConstructor : public ConstructorBase
{
public:
    /// Constructor for the class, needs parameters and a name
    BucketConstructor(const std::string &prefix, /// <Prefix string
                      const BucketConstructorParams &params ); /// <Parameters that define the bucket

    /// Constructs the logical volume
    void Construct();

    /// Returns the Max height above the pmt equator of the bucket
    G4double GetMaxHeight();

    /// Returns the Max depth below the pmt equator of the bucket (note returns a -ive number)
    G4double GetMaxDepth();

    /// Returns the radius of the bucket, for a hex shape this is the distance of the plane that has \n
    /// a normal that passes through the origin (See G4Polyhedra definition).
    G4double GetHexRadius();

    /// Returns the offset from the pmt equator (in Z) where the bucket should be placed
    G4double GetOffset();

    /// Returns the logical volume for the bucket
    G4LogicalVolume* GetLogicalVolume();

protected:
    std::string fPrefix;/// <Prefix string for naming the bucket geometry
    BucketConstructorParams fBucketParameters;/// <Parameters needed to construct
    G4LogicalVolume* fBucketLogic;/// <The completed logical volume

private:
    /// To prevent usage
    BucketConstructor();
};

} // namespace RAT

#endif
\end{Verbatim}

\section{Example Code Definition File}
\label{sec:ExampleDef}
{\bf File:} src/geo/PMT/BucketConstructor.cc
\begin{Verbatim}[gobble=0,numbers=left]
#include <RAT/BucketConstructor.hh>
#include <RAT/Log.hh>

#include <G4Polyhedra.hh>
#include <G4LogicalVolume.hh>
#include <G4VisAttributes.hh>
#include <G4OpticalSurface.hh>
#include <G4SubtractionSolid.hh>
#include <G4Tubs.hh>

namespace RAT
{

BucketConstructor::BucketConstructor(const std::string &prefix,
                                     const BucketConstructorParams &params)
{
    fPrefix = prefix + std::string(``_bucket'');
    fBucketParameters = params;
    fBucketLogic = NULL;
}

void BucketConstructor::Construct()
{
    // Some G4 quirk requires this method
    const G4int nPoints = fBucketParameters.fEdgeZCoord.size();
    G4double *rInner = new G4double[nPoints];
    G4double *zCoord = new G4double[nPoints];
    G4double *rOuter = new G4double[nPoints];

    G4int iLoop;
    for(iLoop=0; iLoop < nPoints; iLoop++) {
        rInner[iLoop] = fBucketParameters.fInnerEdgeRhoCoord[iLoop];
        zCoord[iLoop] = fBucketParameters.fEdgeZCoord[iLoop];
        rOuter[iLoop] = fBucketParameters.fOuterEdgeRhoCoord[iLoop] - 2.0 * PMTGeo::kFaceGap;
    }

    zCoord[0] -= 2.0 * PMTGeo::kFaceGap;
    zCoord[nPoints-1] += 2.0 * PMTGeo::kFaceGap;

    const G4int numHexEdges = 6;
    // Create a hex shell that is the majority of the bucket/panel-sub-part
    G4Polyhedra* bucketOuter = new G4Polyhedra(fPrefix + ``_outer'',
                                               0,
                                               twopi,
                                               numHexEdges,
                                               nPoints,
                                               zCoord,
                                               rInner,
                                               rOuter);
    delete[] rInner;
    delete[] zCoord;
    delete[] rOuter;

    G4VSolid* topHoleSolid = NULL;

    if(fBucketParameters.fTopHexRadius == 0.0) { // Then circular opening
        // Create a solid that represents the top hole (Conc opening)
        topHoleSolid = new G4Tubs(``_top_hole'',
                                  0.0,
                                  fBucketParameters.fTopHoleRadius,
                                  fBucketParameters.fTopHoleDepth,
                                  0.0,
                                  twopi);
    }
    else if( fBucketParameters.fTopHoleRadius == 0.0 ) { // Then hex opening
        G4double zCoord[] = {fBucketParameters.fTopHoleDepth, -fBucketParameters.fTopHoleDepth};
        G4double rInner[] = {0.0, 0.0};
        G4double rOuter[] = {fBucketParameters.fTopHexRadius, fBucketParameters.fTopHexRadius};
        topHoleSolid = new G4Polyhedra(``_top_hole'', M_PI / 18.0 * rad, twopi,
                                       fBucketParameters.fNumPetals,
                                       2,
                                       zCoord,
                                       rInner,
                                       rOuter );
    }
    else
        Log::Die(``BucketConstructor::BucketConstructor: Unclear top opening type.'');

    // Create a solid that represents the bottom hole (PMT base opening)
    G4Tubs *bottomHoleSolid = new G4Tubs(``_bottom_hole'',
                                         0.0,
                                         fBucketParameters.fBottomHoleRadius,
                                         fBucketParameters.fBottomHoleDepth,
                                         0.0,
                                         twopi);

    // Remove the top hole
    G4SubtractionSolid *bucketTopSolid = new G4SubtractionSolid(``_top_temp'',
                                                                bucketOuter,
                                                                topHoleSolid,
                                                                0,
                                                                G4ThreeVector (0, 0, fBucketParameters.fEdgeZCoord.front()));

    // Remove the bottom hole
    G4SubtractionSolid *bucketSolid = new G4SubtractionSolid(fPrefix + ``_solid'',
                                                             bucketTopSolid,
                                                             bottomHoleSolid,
                                                             0,
                                                             G4ThreeVector (0, 0, fBucketParameters.fEdgeZCoord.back()));

    fBucketLogic = new G4LogicalVolume(bucketSolid, fBucketParameters.fBulkMaterial, fPrefix + ``_logic'');

    // Visualization attributes
    if( fBucketParameters.fSimpleVisualisation ) {
        fBucketLogic->SetVisAttributes( G4VisAttributes::Invisible );
    }
    else {
        fBucketLogic->SetVisAttributes( G4Color(0.1,0.1,0.1,1.) );
    }
}

G4LogicalVolume* BucketConstructor::GetLogicalVolume()
{
    if(fBucketLogic == NULL)
        Log::Die(``BucketConstructor::GetLogicalVolume: Construct has not been called.'');
    return fBucketLogic;
}

G4double BucketConstructor::GetMaxHeight()
{
    //Above PMT equator, order of EdgeZCoord matters
    return fBucketParameters.fEdgeZCoord.front() - 2.0 * PMTGeo::kFaceGap;
}

G4double BucketConstructor::GetMaxDepth()
{
    //Below PMT equator, order of EdgeZCoord matters
    return fBucketParameters.fEdgeZCoord.back() + 2.0 * PMTGeo::kFaceGap;
}

G4double BucketConstructor::GetHexRadius()
{
    // Radius is assumed constant
    return fBucketParameters.fOuterEdgeRhoCoord.front() - 2.0 * PMTGeo::kFaceGap;
}

G4double BucketConstructor::GetOffset()
{
    return fBucketParameters.fOffset;
}

} // namespace RAT
\end{Verbatim}

\end{document}

